\documentclass[a4paper,12pt,oneside]{book}
\usepackage[utf8]{inputenc}
\usepackage[T1]{polski}
\usepackage{helvet}
\usepackage{graphicx}
\usepackage{color}
\usepackage{geometry}
\usepackage{fancyhdr}

\fancyfoot[C]{\thepage}
\pagestyle{plain}
\geometry{hmargin={2cm, 2cm}, height=10.0in}
\usepackage{float}
\usepackage{url}
\usepackage{hyperref}

\title{Symulacja komputerowa zaniku sygnału luminescencyjnego w skaleniach}
\author{Olaf Schab}

\begin{document}
\thispagestyle{empty}
\includegraphics[height=37.5mm]{agh_nzw_a_pl_1w_wbr_cmyk.eps}\\
\rule{30mm}{0pt}
{\large\textsf{Wydział Fizyki i Informatyki Stosowanej}}\\
\rule{\textwidth}{3pt}\\
\rule[2ex]
{\textwidth}{1pt}\\
\vspace{7ex}
\begin{center}
{\bf\LARGE\textsf{Praca inżynierska}}\\
\vspace{13ex}
% --------------------------- IMIE I NAZWISKO -------------------------------
{\bf\Large\textsf{Olaf Schab}}\\
\vspace{3ex}
{\sf \small kierunek studiów:} {\bf\small\textsf{informatyka stosowana}}\\
\vspace{15ex}
%% ------------------------ TYTUL PRACY --------------------------------------
{\bf\huge\textsf{Symulacja komputerowa zaniku sygnału luminescencyjnego w skaleniach}}\\
\vspace{14ex}
%% ------------------------ OPIEKUN PRACY ------------------------------------
{\sf \Large Opiekun:} {\bf\Large\textsf{dr inż. Grzegorz Gach}}\\
\vspace{22ex}
\textsf{\bf\large\textsf{Kraków, styczeń 2017}}
\end{center}
%% =====  STRONA TYTUŁOWA PRACY INŻYNIERSKIEJ  ====

\newpage

%% =====  TYŁ STRONY TYTUŁOWEJ PRACY INŻYNIERSKIEJ  ====
{\sf Oświadczam, świadomy(-a) odpowiedzialności karnej za poświadczenie nieprawdy, że niniejszą pracę dyplomową wykonałem(-am) osobiście i samodzielnie i nie korzystałem(-am) ze źródeł innych niż wymienione w pracy.}

\vspace{14ex}

\begin{center}
\begin{tabular}{lr}
~~~~~~~~~~~~~~~~~~~~~~~~~~~~~~~~~~~~~~~~~~~~~~~~~~~~~~~~~~~~~~~~~ &
................................................................. \\
~ & {\sf (czytelny podpis)} \\
\end{tabular}
\end{center}

\newpage
\linespread{1.3}
\selectfont
\newpage
\textbf{Merytoryczna ocena pracy przez opiekuna}
\newpage
\textbf{Merytoryczna ocena pracy przez recenzenta}
\newpage
\vspace*{\fill}

\begin{flushright}
Składam serdeczne podziękowanie
dr inż. Grzegorzowi Gachowi za
cenne wskazówki i pomoc przy
pisaniu niniejszej pracy.
\end{flushright}
\noindent


\vspace{85mm}
\tableofcontents


\chapter{Wstęp}
W archeologii i geologii wiek próbek można określić dzięki wykorzystaniu skaleni. Jest to możliwe dzięki zjawisku stymulowanej luminescencji. Już w latach 60- tych  XX wieku powstał pomysł wykorzystania termoluminescencji, która znalazła wiele zastosowań w archeologii i naukach o Ziemi. Jednym z głównych tematów badań jest datowanie z wykorzystaniem skaleni przy użyciu ich termoluminescencji oraz atermicznym zanikiem luminescencji powodującym zaniżanie daty próbek, który utrudnia datowanie. Atermiczny zanik związany jest z faktem zachodzenia rekombinacji zlokalizowanej, głownie poprzez zachodzące zjawisko tunelowania elektronów z pułapek elektronowych oraz centrów rekombinacyjnych. 

\section{Założenia projektu}
Celem projektu było stworzenie programu symulującego atermiczny zanik sygnału luminescencyjnego w skaleniach. Głównym założeniem było otrzymanie wykresu obrazującego zmianę ilości elektronów znajdujących się w stanie wzbudzonym (znajdujących się w pułapkach) wraz z upływem czasu.

\section{Zjawisko luminescencji w skaleniach}


Jednym ze składników środowiska naturalnego jest promieniowanie jonizujące. Najważniejszym jego źródłem są izotopy promieniotwórcze zawarte w skorupie ziemskiej, atmosferze
i biosferze, emitujące promieniowanie $\alpha$, $\beta$ i $\gamma$. 
W ostatnim okresie, bardzo krótkim w sensie geologicznym oraz historycznym, pojawiły
się nowe jego źródła, związane z działalnością człowieka w zakresie zbrojeń atomowych
i energetyki jądrowej. W dalszym jednak ciągu najważniejszym źródłem promieniowania
jonizującego w środowisku są izotopy promieniotwórcze, które weszły w skład Ziemi w okresie
formowania się układu słonecznego.  Są to przede wszystkim długożyciowe izotopy uranu $U^{238}$, $U^{235}$ i toru $Th^{232}$. Promieniowanie to niesie energię, którą pochłaniają wszystkie substancje występujące
w środowisku - w tym skalenie. Oczywistym jest, że im dłużej substancja  jest poddana promieniowaniu, tym pochłonie jego większą dawkę.

Rzeczywisty kryształ skalenia nie jest idealny. Zawiera on zawsze nieregularności i defekty struktury sieci krystalicznej. Część z nich znajdująca się w paśmie wzbronionym może mieć charakter pułapek, które są zdolne do wychwytywania
elektronów z pasma przewodnictwa i przetrzymywania ich przez długi czas. Stan kryształu, w którym część lub wszystkie pułapki są zapełnione schwytanymi elektronami,
charakteryzuje się nadwyżką energii w porównaniu ze stanem podstawowym.  Nadwyżka ta może zostać wyzwolona i wyemitowana np. w postaci światła. Właśnie to zjawisko wykorzystywane jest jako jedna z metod datowania obiektów archeologicznych
i osadów geologicznych. 
\begin{figure}[h]
\centering
\includegraphics[width=15cm]{strukturapasmowa}
\caption{Struktura pasmowa skalenia. \cite{struktura_pasmowa}}
\label{fig:Struktura pasmowa}
\end{figure}

Zdarza się jednak, że mimo iż uwięziony elektron nie otrzymał dodatkowej energii, będzie w stanie zrekombinować z dziurą. Jest to jest możliwe dzięki  \textbf{zjawisku tunelowania}.

\section{Zjawisko tunelowania}
Tunelowanie, to  proces kwantowo-mechaniczny w którym cząstka ma niezerowe prawdopodobieństwo przejścia przez barierę potencjalną nawet, gdy energia cząstki jest mniejsza od wysokości bariery potencjału.


\begin{figure}[H]
\centering
\includegraphics[width=10cm]{tunelowanie}
\caption{Ilustracja mechanizmu emisji polowej. \cite{tunel_pic}}
\label{fig:Tunelowanie}
\end{figure}



W przypadku skaleni prawdopodobieństwo, że elektron nie przetuneluje do centrum rekombinacji wyrażone jest wzorem:
\cite{wzor}
\begin{equation}
\label{eq:1}
P = e^{\frac{-t}{\tau}}
\end{equation}

\begin{equation}
\label{eq:2}
\tau = S^{-1}e^{\alpha r}
\end{equation}


gdzie:
\begin{itemize}
\item t - czas
\item S - stała częstotliwość prób ucieczki
\item $\alpha$ -  stała zależna od różnicy energii pułapki i dziury
\item r - odległość do przetunelowania
\end{itemize}

Wzór \ref{eq:1} wynika z rozwiązania 1D równania Schrödingera z barierą potencjału.






\chapter{Wybór stosowanych technologii}
\section{Język C++11 oraz wybór kompilatora}
Aplikacja została napisana z użyciem języka C++11. Język ten
wybrano z kilku powodów. Skorzystano tu z jego natury jako języka zorientowanego obiektowo. Dzięki temu podczas wykonania programu stworzono klasy odpowiadające obiektom w świecie realnym - elektrony, pułapki elektronowe itp.
Kolejnym powodem dla którego wykorzystano język C++ jest jego wydajność. Kod napisany
w języku C++ jest kompilowany bezpośrednio i w pełni do kodu maszynowego  wykonywanego przez procesor komputera. Dzięki takiemu
otrzymujemy maksymalną możliwą wydajność inaczej niż gdy używamy innych
obiektowych języków programowania wysokiego poziomu, które są kompilowanye do kodu
zarządzanego (tzw. kodu bajtowego na przykład przypadku Javy) gdzie dodatkowy narzut
generowany jest przez warstwę pośrednią w postaci maszyny wirtualnej. 

Do kompilacji użyto darmowego kompilatora GCC (Gnu Compiler Collecion), konkretnie jego implementacji dla platformy Windows - MinGW w wersji 3.2.2 (Minimalist GNU for Windows).

\section{Środowisko programistyczne}

Program był tworzony z użyciem wieloplatformowego zintegrowanego środowiska programistycznego języków C/C++ - \textbf{CLion} (wersja 2016.2.3) produkcji spółki JetBrains. Było to podyktowane głównie znajomością produktów firmy JetBrains. CLion jest produktem komercyjnym, jednak skorzystano tu z darmowej licencji studenckiej. IDE produkcji JetBrains obsługuje również 	system zarządzania kompilacją \textbf{CMake}, którego główną cechą jest niezależność od używanego kompilatora oraz platformy sprzętowej - CMake nie kompiluje programu samodzielnie, lecz tworzy pliki z regułami kompilacji dla konkretnego środowiska.

\section{System kontroli wersji Git}

W procesie tworzenia aplikacji wykorzystano system kontroli wersji Git oraz darmowy hostingowy serwis internetowy GitHub.
Głównym powodem wykorzystania tego systemu była jego popularność oraz prostota użytkowania.
System kontroli wersji okazał się być niezwykle użytecznym narzędziem pozwalającym na
śledzenie zmian w kodzie, wprowadzanie testowych rozwiązań bez
ryzyka zniszczenia kodu. Cały projekt można pobrać ze strony:
\begin{center}
\url{https://github.com/Sharkuu/Dissertation}
\end{center}

\section{Wizualizacja wyników - Gnuplot}

Po wygenerowaniu danych, aby z wizualizować wykres obrazujący zmianę ilości elektronów znajdujących się w stanie wzbudzonym (znajdujących się w pułapkach) wraz z upływem czasu skorzystano z programu \textbf{Gnuplot}.
\chapter{Implementacja}

Głównym założeniem projektu było stworzenie go w taki sposób, aby kod jak najtrafniej odzwierciedlał rzeczywistość. W tym celu stworzono klasy obiektów reprezentujące rzeczywiste byty w świecie realnym.
\begin{figure}[H]
\centering
\includegraphics[width=15cm]{strukturaprojektu}
\caption{Struktura projektu.}
\label{fig:Struktura projektu}
\end{figure}
\section{Klasy}
Każda z klas posiada własne metody w zależności od swojego przeznaczenia.

Klasa \textit{Electron} odzwierciedla cząstkę elementarną - elektron. Zmiana ilości tych cząstek w stanie wzbudzonym tj. znajdujących się w pułapce jest głównym celem wykonania tej symulacji.

\textit{ElectronHole} jest reprezentacją dziury elektronowej, z którą to elektron po wykorzystaniu zjawiska tunelowego zrekombinuje.

Klasa \textit{Trap} odpowiada defektom w sieci krystalicznej czyli pułapkom, które mogą przechwycić elektron lub dziurę elektronową.
Na potrzeby wykonania tej symulacji założono, że na samym jej początku wszystkie obiekty reprezentujące elektrony oraz dziury elektronowe znajdują się już w pułapkach. Dodatkowo, w symulacji ustalono, że dany elektron jeśli spełni warunek wystąpienia zjawiska tunelowego - po jego zajściu i rekombinacji z dziurą - nie może przetunelować ponownie. Zostało to podyktowane zmniejszeniem oczekiwanego czasu działania programu.

Klasa \textit{Crystal} reprezentuje rzeczywisty kryształ, który będzie poddany symulacji zaniku sygnału luminescencyjnego. Zawiera on w sobie inne obiekty takie jak:

\begin{itemize}
\item Elektrony
\item Dziury elektronowe
\item Obiekty odpowiadające defektom sieci krystalicznej - tzw. pułapki
\end{itemize}

\section{Opis działania programu}
@TODO MOŻLIWE ŻE TU SIE COŚ ZMIENI


W celu otrzymania danych niezbędnych do wygenerowania wykresu zależności między ilością elektronów w pułapkach a upływem czasu uruchomiono stworzony program. Obiekt \textit{Crystal} w swoim konstruktorze generuje podaną ilość dziur elektronowych, pułapek oraz elektronów, a następnie przechowuje je w oddzielnych wektorach/MULTIMAPIE ?!?!?!. Dla każdej z tych cząstek wywoływany jest konstruktor (ustawiający współrzędne położenia) z argumentami, które są losowane z podanego wcześniej przedziału (wartości są podane w Angstremach tj. 1 Å = $10^{-10}$m. Jednostka ta nie jest jednostką układu SI, lecz jest stosowana w fizyce przy opisywaniu obiektów i zjawisk zachodzących w skali atomowej). Jak podkreślono wcześniej, symulacja zakłada, że na jej starcie każdy elektron znajduje się w pułapce. Powoduje to, że para elektron - pułapka ma identyczne współrzędne położenia, ale także obiekt klasy \textit{Trap} przyjmuje wskaźnik na uwięziony w nim elektron.

Następnie za pomocą metody \textit{startSimulation(int time)} obiektu klasy \textit{Crystal} rozpoczynana jest symulacja. Symulowany upływ czasu  zależy od wartości argumentu \textit{time}, który jest wyrażony w dniach tj. wywołanie \emph{startSimulation(365)} oznacza rozpoczęcie działania symulacji symulującej efekt zaniku sygnału luminescencyjnego w czasie 1 roku.


\chapter{Analiza wyników}

Program był tworzony, testowany oraz wykonywany na komputerze posiadającym 4 rdzeniowy procesor Intel Core i7-6700HQ oraz pamięć 8 GB (SO-DIMM DDR4, 2133MHz).


\section{Wygenerowane wykresy}
W celu analizy otrzymanych danych, program został uruchomiony pewną ilość razy, za każdym razem z innymi danymi wejściowymi. Otrzymane wykresy przedstawiają zmianę ilości elektronów znajdujących się w pułapkach w czasie trwania symulacji.

\begin{itemize}

\item Ilość elektronów: $10^{4}$, ilość dziur elektronowych: $10^{4}$, zakres w którym losowane są położenia cząstek: [-800,800], symulowany czas: 1000 dni.
\begin{figure}[H]
\centering
\includegraphics[width=17cm]{wykres2}
\caption{ Czas wykonywania symulacji: xx}
\label{fig:Tunelowanie}
\end{figure}

\item Ilość elektronów: $10^{4}$, ilość dziur elektronowych: $10^{4}$, zakres w którym losowane są położenia cząstek: [-900,900], symulowany czas: 5 lat.
\begin{figure}[H]
\centering
\includegraphics[width=17cm]{wykres2}
\caption{ Czas wykonywania symulacji: 13h}
\label{fig:Tunelowanie}
\end{figure}

\item Ilość elektronów: $10^{3}$, ilość dziur elektronowych: $10^{3}$, zakres w którym losowane są położenia cząstek: [-500,500], symulowany czas: 1000 dni.
\begin{figure}[H]
\centering
\includegraphics[width=17cm]{wykres2}
\caption{ Czas wykonywania symulacji: xx}
\label{fig:Tunelowanie}
\end{figure}

\item Ilość elektronów: $10^{2}$, ilość dziur elektronowych: $10^{2}$, zakres w którym losowane są położenia cząstek: [-400,400], symulowany czas: 1000 dni.
\begin{figure}[H]
\centering
\includegraphics[width=17cm]{wykres2}
\caption{ Czas wykonywania symulacji: xx}
\label{fig:Tunelowanie}
\end{figure}

\item Ilość elektronów: $10^{2}$, ilość dziur elektronowych: $10^{2}$, zakres w którym losowane są położenia cząstek: [-200,200], symulowany czas: 1000 dni.
\begin{figure}[H]
\centering
\includegraphics[width=17cm]{wykres2}
\caption{ Czas wykonywania symulacji: xx}
\label{fig:Tunelowanie}
\end{figure}




\end{itemize}
\section{Analiza wykresów}
Analizując otrzymane wykresy można zauważyć znaczny spadek ilości elektronów w pułapkach na samym początku wykonywania programu. Jest to spowodowane faktem, że wszystkie współrzędne cząstek są losowane z podanego zakresu. Oznacza to, że na początku symulacji wiele pułapek elektronowych zawierających elektrony znajduje się bardzo blisko centr rekombinacji. Powołując się na równania \ref{eq:1} oraz \ref{eq:2} obserwujemy, że odległość między pułapką a centrum rekombinacji znacząco wpływa na prawdopodobieństwo zajścia efektu tunelowego, co skutkuje zaobserwowanym początkowym spadkiem ilości wzbudzonych elektronów. Po wystąpieniu znaczącego spadku można zaobserwować, że ilość elektronów znajdujących się w pułapkach zaczyna maleć z dużo mniejszą częstotliwością. Wykres zaczyna wtedy przypominać funkcję liniową. Obrazuje to fakt występowania atermicznego zaniku luminescencyjnego, który zachodzi samoistnie dzięki zjawisku tunelowania, bez dostarczania dodatkowej energii. 

Ważnym parametrem symulacji jest zakres z którego losowane są wartości położenia dla cząstek. Porównując ze sobą wykres \ref{asd} oraz \ref{asd} gdzie ilość elektronów oraz symulowany czas są takie same, a zakres wartości położeń cząstki znacząco się różni, zaobserwowano, że im większa gęstość ładunków tym efektu tunelowania zachodzi szybciej, co jest zgodne z przewidywaniami wynikającymi ze wzoru \ref{eq:2}.

\include{wnioski}

\begin{thebibliography}{wstep}

\bibitem{struktura_pasmowa}
  \emph{Instrukcja do ćwiczenia laboratoryjnego z dozymetrii promieniowania jonizującego dla
studentów specjalności Fizyka Medyczna i pokrewnych}, WFiIS AGH, Kraków 1993, s.3 rys. 1
\bibitem{tunel_pic}
Piotr Psuja,
\emph{Właściwosci luminescencyjne i katodoluminescencyjne
nanometrycznych kompozytów ITO (In2O3/SnO2)
domieszkowanych jonami ziem rzadkich}, Wrocław 2009, s.14 rys. 3.1
\bibitem{wzor}
D.J. Huntley,
\emph{An explanation of the power-law decay of
luminescence}, Styczeń 2006, s.1360 równanie 1



\end{thebibliography}

\end{document}