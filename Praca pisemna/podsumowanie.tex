\chapter{Podsumowanie}

Celem wykonywanej pracy było napisanie programu symulującego atermiczny zanik sygnału luminescencyjnego w skaleniach. Dzięki stworzeniu klas odpowiadającym rzeczywistym cząstkom uzyskano program założona podstawowa funkcjonalność została osiągnięta, a zaimplementowany projekt jest prosty do zrozumienia oraz dalszego rozwoju. 

Praca została usprawniona dzięki skorzystaniu z profesjonalnego IDE CLion, oraz innym \hyperref[tech:all]{narzędziom} użytym w trakcie implementacji. Dzięki nim, stworzenie przejrzystego kodu wraz z dokumentacja nie zawierającego błędów było o wiele prostsze i zajęło mniej czasu.

Używając programu należy pamiętać, że symuluje on proces długotrwały. W rzeczywistości elektrony znajdujące się w pułapkach mogą być tam przechowywane przez setki lat. Symulacja takiego zjawiska nie jest prosta, gdyż wymaga bardzo dużej ilości obliczeń, co znacznie wydłuża czas potrzebny do otrzymania danych. W celu wykonania symulacji w rozsądnym czasie w  prezentowanych wynikach ilość cząstek nie przekroczyła rzędu $10^{4}$ \footnote[7]{W rzeczywistym krysztale skalenia cząstek może być znacznie więcej}, a symulowany czas wynosił najczęściej zaledwie 1000 dni. Pozwoliło to na przeprowadzenie symulacji trwającej maksymalnie około 24 godzin. Czas wykonywania symulacji można zmniejszyć optymalizując kod na przykład używając drzewa \emph{kd}, jak wspomniano w rozdziale \hyperref[rozwoj:1]{5}. 
