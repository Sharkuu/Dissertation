\chapter{Analiza wyników}

Program był tworzony, testowany oraz wykonywany na komputerze o podanej specjalizacji:
\begin{itemize}
\item Procesor: Intel Core i7-6700HQ (4 rdzenie)
\item Pamięć: 8 GB (SO-DIMM DDR4, 2133MHz)
\end{itemize}

\section{Wygenerowane wykresy}
W celu analizy otrzymanych danych, program został uruchomiony pewną ilość razy, za każdym razem z innymi danymi wejściowymi. Starano się tak dopasować ich wartości, aby gęstość rozkładu cząstek była stała.

------------------

TU WYKRESY Z PODPISAMI JAKIE BYŁY ARGUENTY, CZAS WYKONANIA ,ILE PAMIECI UZYWALO(?)

---------
\section{Analiza wykresów}
Analizując otrzymane wykresy można zauważyć znaczny spadek ilości elektronów w pułapkach na samym początku wykonywania programu. Jest to spowodowane tym, że wszystkie współrzędne cząstek są losowane z podanego zakresu. Oznacza to, że na początku symulacji wiele pułapek elektronowych zawierających elektrony znajduje się bardzo blisko centrów rekombinacji. Jak widać w równaniach (1.1) oraz (1.2) odległość między nimi znacząco wpływa na prawdopodobieństwo zajścia efektu tunelowego, stąd też wynika zaobserwowany znaczący spadek ilości wzbudzonych elektronów.