\chapter{Wybór stosowanych technologii}
\section{Język C++11 oraz wybór kompilatora}
Aplikacja została napisana z użyciem języka C++11. Język ten
wybrano z kilku powodów. Skorzystano tu z jego natury jako języka zorientowanego obiektowo. Dzięki temu podczas wykonania programu stworzono klasy odpowiadające obiektom w świecie realnym - elektrony, pułapki elektronowe itp.
Kolejnym powodem dla którego wykorzystano język C++ jest jego wydajność. Kod napisany
w języku C++ jest kompilowany bezpośrednio i w pełni do kodu maszynowego  wykonywanego przez procesor komputera. Dzięki takiemu
otrzymujemy maksymalną możliwą wydajność inaczej niż gdy używamy innych
obiektowych języków programowania wysokiego poziomu, które są kompilowanye do kodu
zarządzanego (tzw. kodu bajtowego na przykład przypadku Javy) gdzie dodatkowy narzut
generowany jest przez warstwę pośrednią w postaci maszyny wirtualnej. 

Do kompilacji użyto darmowego kompilatora GCC (Gnu Compiler Collecion), konkretnie jego implementacji dla platformy Windows - MinGW w wersji 3.2.2 (Minimalist GNU for Windows).

\section{Środowisko programistyczne}

Program był tworzony z użyciem wieloplatformowego zintegrowanego środowiska programistycznego języków C/C++ - \textbf{CLion} (wersja 2016.2.3) produkcji spółki JetBrains. Było to podyktowane głównie znajomością produktów firmy JetBrains. CLion jest produktem komercyjnym, jednak skorzystano tu z darmowej licencji studenckiej. IDE produkcji JetBrains obsługuje również 	system zarządzania kompilacją \textbf{CMake}, którego główną cechą jest niezależność od używanego kompilatora oraz platformy sprzętowej - CMake nie kompiluje programu samodzielnie, lecz tworzy pliki z regułami kompilacji dla konkretnego środowiska.

\section{System kontroli wersji Git}

W procesie tworzenia aplikacji wykorzystano system kontroli wersji Git oraz darmowy hostingowy serwis internetowy GitHub.
Głównym powodem wykorzystania tego systemu była jego popularność oraz prostota użytkowania.
System kontroli wersji okazał się być niezwykle użytecznym narzędziem pozwalającym na
śledzenie zmian w kodzie, wprowadzanie testowych rozwiązań bez
ryzyka zniszczenia kodu. Cały projekt można pobrać ze strony:
\begin{center}
\url{https://github.com/Sharkuu/Dissertation}
\end{center}

\section{Wizualizacja wyników - Gnuplot}

Po wygenerowaniu danych, aby z wizualizować wykres obrazujący zmianę ilości elektronów znajdujących się w stanie wzbudzonym (znajdujących się w pułapkach) wraz z upływem czasu skorzystano z programu \textbf{Gnuplot}.